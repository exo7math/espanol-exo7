%%%%%%%%%%%%%%%%%% PREAMBULE %%%%%%%%%%%%%%%%%%

\documentclass[12pt,a4paper]{article}

\usepackage{amsfonts,amsmath,amssymb,amsthm}
\usepackage[francais]{babel}
\usepackage[utf8]{inputenc}
\usepackage[T1]{fontenc}

%----- Ensemles : entiers, reels, complexes -----
\newcommand{\Nn}{\mathbb{N}} \newcommand{\N}{\mathbb{N}}
\newcommand{\Zz}{\mathbb{Z}} \newcommand{\Z}{\mathbb{Z}}
\newcommand{\Qq}{\mathbb{Q}} \newcommand{\Q}{\mathbb{Q}}
\newcommand{\Rr}{\mathbb{R}} \newcommand{\R}{\mathbb{R}}
\newcommand{\Cc}{\mathbb{C}} \newcommand{\C}{\mathbb{C}}

%----- Modifications de symboles -----
\renewcommand {\epsilon}{\varepsilon}
\renewcommand {\Re}{\mathop{\mathrm{Re}}\nolimits}
\renewcommand {\Im}{\mathop{\mathrm{Im}}\nolimits}

%----- Fonctions usuelles -----
\newcommand{\ch}{\mathop{\mathrm{ch}}\nolimits}
\newcommand{\sh}{\mathop{\mathrm{sh}}\nolimits}
\renewcommand{\tanh}{\mathop{\mathrm{th}}\nolimits}
\newcommand{\Arcsin}{\mathop{\mathrm{Arcsin}}\nolimits}
\newcommand{\Arccos}{\mathop{\mathrm{Arccos}}\nolimits}
\newcommand{\Arctan}{\mathop{\mathrm{Arctan}}\nolimits}
\newcommand{\Argsh}{\mathop{\mathrm{Argsh}}\nolimits}
\newcommand{\Argch}{\mathop{\mathrm{Argch}}\nolimits}
\newcommand{\Argth}{\mathop{\mathrm{Argth}}\nolimits}
\newcommand{\pgcd}{\mathop{\mathrm{pgcd}}\nolimits} 

%----- Commandes special dessin a ajouter localement ------
\usepackage{geometry}
\usepackage{pstricks}
\usepackage{pst-plot}
\usepackage{pst-node}
\usepackage{graphics,epsfig}

\newcmykcolor{DarkGreen}{0.4 0 0.76 0.3}
\newcmykcolor{Goldenrod}{0 0.1 0.84 0}
\newcmykcolor{Fuchsia}{0.47 0.91 0 0.08}

\pagestyle{empty}

% Que faire avec ce fichier monimage.tex ?
%   1/ latex monimage.tex
%   2/ dvips monimage.dvi
%   3/ ps2eps monimage.ps
%   4/ ps2pdf -dEPSCrop monimage.eps
%   5/ Dans votre fichier d'exos \includegraphics{monimage}

\begin{document}

\begin{center}
\begin{pspicture}(-5,-4)(8,6)
\psplot[linecolor=blue]{0.35}{8}{2 x div}
\psplot[linecolor=blue]{-5}{-0.6}{2 x div}
\psdots[linecolor=blue](-2,-1)(2,1)
\uput[d](-2,-1){\textcolor{blue}{$Q$}}
\uput[u](2,1){\textcolor{blue}{$P$}}
\pscircle[linecolor=blue](2,1){4.472}
\psline[linecolor=blue,linestyle=dashed](-5,0)(8,0)
\psline[linecolor=blue,linestyle=dashed](0,-4)(0,6)
\psdots[linecolor=red](-0.8,-2.45)(6.42,0.34)(0.4,5.17)
\pspolygon[linecolor=red](-0.8,-2.45)(6.42,0.34)(0.4,5.17)
\uput[ur](6.42,0.34){\textcolor{red}{$M_1$}}
\uput[ur](0.4,5.17){\textcolor{red}{$M_2$}}
\uput[dl](-0.8,-2.45){\textcolor{red}{$M_3$}}
\uput[d](-4.5,-0.5){\textcolor{blue}{$(\mathcal{H})$}}
\uput[ur](4,4){\textcolor{blue}{$(\mathcal{C})$}}
\end{pspicture}
\end{center}


\end{document}
